\documentclass[12pt]{article} 
\usepackage[papersize={8.5in,11in}, margin=1in]{geometry}
\usepackage[utf8]{inputenc}
\usepackage[french]{babel}
\usepackage[T1]{fontenc}
\usepackage{lmodern}
\usepackage{fourier} % Use the Adobe Utopia font for the document - comment this line to return to the LaTeX 
\usepackage{amsmath,amsfonts,amsthm} % Math packages
\usepackage{float}
\usepackage{sectsty} % Allows customizing section commands
\usepackage{placeins}
\usepackage{cancel}
%\usepackage{amssymb}
\usepackage{amsbsy}
\usepackage{amsmath,amsfonts,amssymb}
%\allsectionsfont{\normalfont}

\usepackage{graphicx}
\graphicspath{{Figures/}}

%\usepackage{fancyhdr} % Custom headers and footers
%\pagestyle{fancyplain} % Makes all pages in the document conform to the custom headers and footers
%\fancyhead{} % No page header - if you want one, create it in the same way as the footers below
%\fancyfoot[L]{} % Empty left footer
%\fancyfoot[C]{} % Empty center footer
%\fancyfoot[R]{\thepage} % Page numbering for right footer
%\renewcommand{\headrulewidth}{0pt} % Remove header underlines
%\renewcommand{\footrulewidth}{0pt} % Remove footer underlines
\setlength{\headheight}{5pt} % Customize the height of the header
\numberwithin{figure}{section} % Number figures within sections (i.e. 1.1, 1.2, 2.1, 2.2 instead of 1, 2, 3, 4)
\numberwithin{table}{section} % Number tables within sections (i.e. 1.1, 1.2, 2.1, 2.2 instead of 1, 2, 3, 4)
\setlength\parindent{0pt}


\def\FIG[#1]#2#3#4{		
	\begin{figure}[htb]		
		\begin{center}			
			\includegraphics[#1]{#2}
		\end{center}			
		\caption{#3}			
		\label{#4}			
	\end{figure}			
}	

\newcommand{\horrule}[1]{\rule{\linewidth}{#1}} 
\newcommand{\impl}{$\Rightarrow$}
\usepackage{pdfpages}


\newcommand{\ClassName}{RobocupUlaval }
\newcommand{\ClassNumber}{ULtron}
\newcommand{\Subject}{Modélisation et contrôle}

\newcommand{\vect}[1]{\boldsymbol{#1}}

%----------------------------------------------------------------------------------------

\begin{document}

\begin{center}
\normalfont \normalsize 
\huge \Subject\\
\horrule{0.5pt} \\[0.4cm]
\huge \ClassName - \ClassNumber \\ 
\horrule{0.5pt} \\[0.4cm]
\large \textit{Par: Simon Bouchard}\\
\textit{\today}
\end{center}

% \FIG[width=0.75\paperwidth]{filename}{caption}{fig:refname}

%---------------------------------------------------------------------------------------

\section{Modèle du robot}

\subsection{Équation en continu}
\[
	\vect{\dot{\omega}} = \vect{M_1}\vect{u} - \vect{M_2}\vect{\omega}
\]
où,
\begin{itemize}
	\item[$\vect{\omega}$] est le vecteur de vitesses angulaires des roues en rad/s;
	\item[$\vect{u}$] est le vecteur de rapport cyclique de commande aux moteurs.
\end{itemize}

\subsection{Valeurs numériques}
\[
M_1(0) = \begin{bmatrix}
  520.2 & 346.8 & 173.4 & 346.8 \\[0.3em]
  346.8 & 520.2 & 346.8 & 173.4 \\[0.3em]
  173.4 & 346.8 & 520.2 & 346.8 \\[0.3em]
  346.8 & 173.4 & 346.8 & 520.2
     \end{bmatrix}
\]

\[
M_2(0) = \begin{bmatrix}
    4.3350  &  2.8900  &  1.4450  &  2.8900 \\[0.3em]
    2.8900  &  4.3350  &  2.8900  &  1.4450 \\[0.3em]
    1.4450  &  2.8900  &  4.3350  &  2.8900 \\[0.3em]
    2.8900  &  1.4450  &  2.8900  &  4.3350
     \end{bmatrix}
\]

\[
M_{1opt} = \begin{bmatrix}
		832.5160 & 348.0510 & 150.7291 & 327.6491 \\[0.3em]
        486.8178 & 692.4930 & 399.4206 & 117.5622 \\[0.3em]
        241.0265 & 239.6313 & 761.3489 & 235.5609 \\[0.3em]
        221.2959 &  59.1184 & 116.8171 & 702.2684
     \end{bmatrix}
\]

\[
M_{2opt} = \begin{bmatrix}
        4.5265  &  0.2949  &  1.1064  &  0.3471 \\[0.3em]
        1.0369  &  2.9667  &  0.8380  &  0.1454 \\[0.3em]
        0.1446  &  0.2892  &  2.9959  &  0.2894 \\[0.3em]
        0.7098  &  0.1451  &  0.2896  &  3.3578
     \end{bmatrix}
\]


\section{Loi de commande du robot}

\subsection{NRMC continu}

\[
	\vect{u} = \vect{M_1^{-1}}\left( \vect{\Lambda_{\omega}}\vect{\omega_m} - \vect{\Lambda_{\omega}}\vect{r_w} +\vect{M_2\omega} +\vect{K_pe + K_Ie_I} \right)
\]
\[
	\vect{\dot{\omega}_m} = \vect{\Lambda_{\omega}}\vect{\omega_m} - \vect{\Lambda_{\omega}}\vect{r_w} \quad \Rightarrow \quad  \frac{\omega_{mi}}{r_{\omega i}} = \frac{1}{s+\lambda_i}
\]
où,
\begin{itemize}
	\item[$\vect{e} = \vect{\omega_m - \omega}$];
	\item[$\vect{e_I} = \int \vect{e} dt$];
	\item[$\vect{K_p}$] est le gain proportionnel;
	\item[$\vect{K_I}$]	est le gain intégral;
	\item[$\vect{\omega_m}$] est la consigne filtrée de vitesse angulaire;
	\item[$\vect{r_w}$] est la consigne de vitesse angulaire.
\end{itemize}
\end{document}
